% !TEX TS-program = pdflatex
% !TEX encoding = UTF-8 Unicode

% This is a simple template for a LaTeX document using the "article" class.
% See "book", "report", "letter" for other types of document.

\documentclass[11pt]{article} % use larger type; default would be 10pt

\usepackage[utf8]{inputenc} % set input encoding (not needed with XeLaTeX)
\usepackage{url}

%%% PAGE DIMENSIONS
\usepackage{geometry} % to change the page dimensions
\geometry{a4paper} % or letterpaper (US) or a5paper or....

\usepackage{graphicx} % support the \includegraphics command and options

%%% PACKAGES
\usepackage{booktabs} % for much better looking tables
\usepackage{array} % for better arrays (eg matrices) in maths
\usepackage{paralist} % very flexible & customisable lists (eg. enumerate/itemize, etc.)
\usepackage{verbatim} % adds environment for commenting out blocks of text & for better verbatim
\usepackage{subfig} % make it possible to include more than one captioned figure/table in a single float
% These packages are all incorporated in the memoir class to one degree or another...

%%% HEADERS & FOOTERS
\usepackage{fancyhdr} % This should be set AFTER setting up the page geometry
\pagestyle{fancy} % options: empty , plain , fancy
\renewcommand{\headrulewidth}{0pt} % customise the layout...
%\lhead{}\chead{}\rhead{}
%\lfoot{}\cfoot{\thepage}\rfoot{}

%%% SECTION TITLE APPEARANCE
%%%\usepackage{sectsty}
%%%\allsectionsfont{\sffamily\mdseries\upshape} % (See the fntguide.pdf for font help)
% (This matches ConTeXt defaults)

%%% ToC (table of contents) APPEARANCE
\usepackage[nottoc,notlof,notlot]{tocbibind} % Put the bibliography in the ToC
\usepackage[titles,subfigure]{tocloft} % Alter the style of the Table of Contents
\renewcommand{\cftsecfont}{\rmfamily\mdseries\upshape}
\renewcommand{\cftsecpagefont}{\rmfamily\mdseries\upshape} % No bold!

%%% END Article customizations

%%% The "real" document content comes below...

\title{Cybercrime Science Research topic: Trends in malware development}
\author{Mark Vijfvinkel \& Aram Verstegen \\ 4077148 4092368}
\date{} % Activate to display a given date or no date (if empty),
         % otherwise the current date is printed 

\begin{document}
\maketitle

\abstract{
%We aim to investigate the current trends in malware.
%Focussing on the past five years, we will try to map out the various propagation methods, features, protocols and economics of various malware such as banking trojans, Remote File Inclusion bots, scare- and ransomware.
%By collecting various types of current malware, we will try to find patterns of malware operation and classify them as a means to predict coming trends.
%
%We hope to be able to discover more recent malware during the course of the semester, which will hopefully bring even more variation to the already diverse ecosystem of malware.
%One of the products we hope to generate as an offshoot of our research is a visual representation of the malware trends over time.


%\cite{kanich2008spamalytics}
}

\section{Problem}
Malware (a conjunction of Malicious Software) is a term to cover various families of malicious software.
In such families we can distinguish malware by their method of spreading, like (self-replicating) \textit{viruses} - or \textit{worms} - and \textit{trojans} (innocent-looking programs containing malicious payload).
We can also identify a distinction between types of payloads, with illustrative titles like \textit{keyloggers}, \textit{adware} and \textit{spyware}.
Viruses that initially manifested as a general annoyance by blatantly drawing attention to themselves have since become ever more advanced at remaining stealthy and, for example, surreptitiously exfiltrating the home user's internet banking details in a completely automated way.
The most insidious of such malware would be the \textit{bot} (short for robot), a program that provides control over the machine to a remote attacker's Command and Control (C&C) software, by which it's possible to create a \textit{botnet} and is often used to stage a \textit{Distributed Denial of Service} (DDoS) attack.

%Most users have been exposed to malware at some point.
Modern malware has an impact on both home and business users, and has become a persistent problem since the worldwide adoption of the internet.
Internet banking fraud is a serious problem today, and the even more insidious threat of identity theft is on the rise. (TODO citation)

Given that home users have become a target for criminals, it is clear that companies, by having a much bigger public presence, are a more attractive target for attacks over the internet.
Financial damages are incurred due to downtime in nearly all cases, while in the case of malware that is often only a side effect of mitigating the infection itself, rather than a direct result of it.
Next to the usual threats of identity theft and stolen login information, corporate networks are also at risk of losing control over their trade secrets.
Although attacks on corporate networks can be expected to use more sophisticated - meaning: not completely automated - methods to gain and maintain access than what the home user is exposed to, the completely automated malware indiscriminately targets home and business users alike.

%For example, it is possible to buy or rent a botnet to send spam or attack a website with a DDoS attack.
%Some providers of such DDoS services even provide a free trial in the form of a 5-minute DDoS attack as a means to establish legitimacy.

Anti-Virus (AV) programs are advertised as a treatment against malware.
By scanning files on the computer's hard drive and comparing them to signatures of known malware or doing heuristic analysis of the code, it is possible to detect and remove infections.

Research has shown that criminal organizations actively develop malware for a specific purpose.
These darker shades of security research employ techniques of obfuscation to remain undetected from such as self-modifying code, run-time decryption of code, injection of code into other processes and the like, albeit for different end goals.
A common practice in virus spreading is the use of a so-called `packer' or `crypter', software that encapsulates another program in an obfuscated way by loading the latter through a unique (and therefore unbeknownst to Anti-Virus vendors) `stub' of loader code.
This provides a modular way to employ (and re-use) different types of malware without triggering detection by AV software.

Malware authors and AV vendors are in a continual arms race to outsmart one another, and must monitor each others development attentively to remain effective.
Our research focusses on the competence of AV vendors to keep up with the steady stream of new crypters that are published freely on the internet.
Using a crypter allows us to create a harmless packed binary, which we can then test for detection in various AV packages to assess the AV vendors' vigilance in detecting new crypters and distributing appropriate signatures.

%A tool for malware developers to beat the anti-virus programs is the use of crypters.
%A crypter is software designed to make other software undetectable for an anti-virus program by encrypting it. Therefore it becomes possible to use viruses that would normally be detected. 

%The use of crypters would enable criminal organizations to re-use malware for their financial gain.

 
\subsection{Research question}
\subsubsection{Sub-questions}
\subsubsection{Hypothesis}
\section{Scope}
\section{Motivation}
\section{Methodology}
\section{Time schedule}

Source language
Lines of code / binary size
Encryption methods
Country of origin
Motivation
Target Operating System / execution environment
Number of known variants


\begin{itemize}
\item{Remote Administration Tools/Trojans (RATs)}
\item{Fake Antivirus}
\item{Fake Antivirus}
\end{itemize}

\bibliographystyle{alpha}
%\bibliography{references}
\end{document}
