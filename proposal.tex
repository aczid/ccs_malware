% !TEX TS-program = pdflatex
% !TEX encoding = UTF-8 Unicode

% This is a simple template for a LaTeX document using the "article" class.
% See "book", "report", "letter" for other types of document.

\documentclass[11pt]{article} % use larger type; default would be 10pt

\usepackage[utf8]{inputenc} % set input encoding (not needed with XeLaTeX)
\usepackage{url}

%%% PAGE DIMENSIONS
\usepackage{geometry} % to change the page dimensions
\geometry{a4paper} % or letterpaper (US) or a5paper or....

\usepackage{graphicx} % support the \includegraphics command and options

%%% PACKAGES
\usepackage{booktabs} % for much better looking tables
\usepackage{array} % for better arrays (eg matrices) in maths
\usepackage{paralist} % very flexible & customisable lists (eg. enumerate/itemize, etc.)
\usepackage{verbatim} % adds environment for commenting out blocks of text & for better verbatim
\usepackage{subfig} % make it possible to include more than one captioned figure/table in a single float
% These packages are all incorporated in the memoir class to one degree or another...

%%% HEADERS & FOOTERS
\usepackage{fancyhdr} % This should be set AFTER setting up the page geometry
\pagestyle{fancy} % options: empty , plain , fancy
\renewcommand{\headrulewidth}{0pt} % customise the layout...
%\lhead{}\chead{}\rhead{}
%\lfoot{}\cfoot{\thepage}\rfoot{}

%%% SECTION TITLE APPEARANCE
%%%\usepackage{sectsty}
%%%\allsectionsfont{\sffamily\mdseries\upshape} % (See the fntguide.pdf for font help)
% (This matches ConTeXt defaults)

%%% ToC (table of contents) APPEARANCE
\usepackage[nottoc,notlof,notlot]{tocbibind} % Put the bibliography in the ToC
\usepackage[titles,subfigure]{tocloft} % Alter the style of the Table of Contents
\renewcommand{\cftsecfont}{\rmfamily\mdseries\upshape}
\renewcommand{\cftsecpagefont}{\rmfamily\mdseries\upshape} % No bold!

%%% END Article customizations

%%% The "real" document content comes below...

\title{Cybercrime Science Research topic: Trends in malware development}
\author{Mark Vijfvinkel \& Aram Verstegen \\ 4077148 4092368}
\date{} % Activate to display a given date or no date (if empty),
         % otherwise the current date is printed 

\begin{document}
\maketitle

\abstract{
%We aim to investigate the current trends in malware.
%Focussing on the past five years, we will try to map out the various propagation methods, features, protocols and economics of various malware such as banking trojans, Remote File Inclusion bots, scare- and ransomware.
%By collecting various types of current malware, we will try to find patterns of malware operation and classify them as a means to predict coming trends.
%
%We hope to be able to discover more recent malware during the course of the semester, which will hopefully bring even more variation to the already diverse ecosystem of malware.
%One of the products we hope to generate as an offshoot of our research is a visual representation of the malware trends over time.


%\cite{kanich2008spamalytics}
}

\section{Problem}
Malware (Malicious Software) have been around almost as long as there have been computers and come in various forms e.g. viruses, trojans, Remote File Inclusion bots, keyloggers and spyware. These types of malware have had a worldwide impact on the home user, businesses and have become a resistant problem since internet connectivity increased. For the home user malware started as an annoying pest, but since internet banking has increased, home users have become a target for criminals. Companies have bigger problems when it comes to malware e.g. downtime, data theft, backdoors, identity theft and spam, which eventually results in financial damage.
Research has also shown that criminal organizations actively develop malware for a specific purpose. For example, it is possible to buy or rent a botnet (a group infected computers under the control a person) to send spam or attack a website with a Denial of Service attack.

Anti-virus programs have been advertised as a treatment against malware. By scanning the files on the computer and comparing them to the signature of the malware it is possible to detect and remove it. Since the development of malware and anti-virus programs it has been an arms race between them to outsmart each other. A tool for malware developers to beat the anti-virus programs is the use of crypters. A crypter is software designed to make other software undetectable for an anti-virus program by encrypting it. Therefore it becomes possible to use viruses that would normally be detected. 

The use of crypters would enable criminal organizations to re-use malware for their financial gain.

 
\subsection{Research question}
We aim to answer to the following research question:

What is the effectiveness of anti-virus programs against the use of so called "crypters" on current software?

\subsubsection{Subquestions}

What are the principles behind modern anti-virus?
What are the features in crimeware and how is this applied?

\subsubsection{Hypothesis}

We expect that anti-virus programs will detect the use of a crypter on software after three days. 

\section{Scope}
\section{Motivation}
\section{Methodology}



\section{Time schedule}

Source language
Lines of code / binary size
Encryption methods
Country of origin
Motivation
Target Operating System / execution environment
Number of known variants


\begin{itemize}
\item{Remote Administration Tools/Trojans (RATs)}
\item{Fake Antivirus}
\item{Fake Antivirus}
\end{itemize}

\bibliographystyle{alpha}
%\bibliography{references}
\end{document}
