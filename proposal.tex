% !TEX TS-program = pdflatex
% !TEX encoding = UTF-8 Unicode

% This is a simple template for a LaTeX document using the "article" class.
% See "book", "report", "letter" for other types of document.

\documentclass[11pt]{article} % use larger type; default would be 10pt

\usepackage[utf8]{inputenc} % set input encoding (not needed with XeLaTeX)
\usepackage{url}

%%% PAGE DIMENSIONS
\usepackage{geometry} % to change the page dimensions
\geometry{a4paper} % or letterpaper (US) or a5paper or....

\usepackage{graphicx} % support the \includegraphics command and options

%%% PACKAGES
\usepackage{booktabs} % for much better looking tables
\usepackage{array} % for better arrays (eg matrices) in maths
\usepackage{paralist} % very flexible & customisable lists (eg. enumerate/itemize, etc.)
\usepackage{verbatim} % adds environment for commenting out blocks of text & for better verbatim
\usepackage{subfig} % make it possible to include more than one captioned figure/table in a single float
% These packages are all incorporated in the memoir class to one degree or another...

%%% HEADERS & FOOTERS
\usepackage{fancyhdr} % This should be set AFTER setting up the page geometry
\pagestyle{fancy} % options: empty , plain , fancy
\renewcommand{\headrulewidth}{0pt} % customise the layout...
%\lhead{}\chead{}\rhead{}
%\lfoot{}\cfoot{\thepage}\rfoot{}

%%% SECTION TITLE APPEARANCE
%%%\usepackage{sectsty}
%%%\allsectionsfont{\sffamily\mdseries\upshape} % (See the fntguide.pdf for font help)
% (This matches ConTeXt defaults)

%%% ToC (table of contents) APPEARANCE
\usepackage[nottoc,notlof,notlot]{tocbibind} % Put the bibliography in the ToC
\usepackage[titles,subfigure]{tocloft} % Alter the style of the Table of Contents
\renewcommand{\cftsecfont}{\rmfamily\mdseries\upshape}
\renewcommand{\cftsecpagefont}{\rmfamily\mdseries\upshape} % No bold!

%%% END Article customizations

%%% The "real" document content comes below...

\title{Cybercrime Science: effectiveness of Anti-Virus vendors in defending against previously unidentified threats}
\author{Mark Vijfvinkel \& Aram Verstegen \\ 4077148 4092368}
\date{} % Activate to display a given date or no date (if empty),
         % otherwise the current date is printed 

\begin{document}
\maketitle

%\abstract{
%We aim to investigate the current trends in malware.
%Focussing on the past five years, we will try to map out the various propagation methods, features, protocols and economics of various malware such as banking trojans, Remote File Inclusion bots, scare- and ransomware.
%By collecting various types of current malware, we will try to find patterns of malware operation and classify them as a means to predict coming trends.
%
%We hope to be able to discover more recent malware during the course of the semester, which will hopefully bring even more variation to the already diverse ecosystem of malware.
%One of the products we hope to generate as an offshoot of our research is a visual representation of the malware trends over time.


%\cite{kanich2008spamalytics}
%}

\section{Problem}
The conjunction `malware' is an umbrella term to cover various families of Malicious Software.
We can classify malware by its means of spreading, like (self-replicating) \textit{viruses} - or \textit{worms} - and \textit{trojans} (innocent-looking programs containing malicious payload).
We can also identify a distinction between types of payloads, with illustrative titles like \textit{keyloggers}, \textit{adware} and \textit{spyware}.

Viruses initially manifested as a general annoyance by blatantly drawing attention to themselves but have since evolved to become ever more advanced at remaining stealthy and, for example, surreptitiously exfiltrating the home user's internet banking details in a completely automated way.
The most insidious of such malware would be the \textit{bot} (short for robot), a program that provides control over the machine to a remote attacker's Command and Control (C\&C) software, with which it is possible to create a \textit{botnet}, often used to stage a \textit{Distributed Denial of Service} (DDoS) attack.

%Most users have been exposed to malware at some point.
Modern malware has an impact on both home and business users, and has become a persistent problem since the worldwide adoption of the internet.
Internet banking fraud is a serious problem today, and the even more insidious threat of identity theft is on the rise. (TODO citation)

% NOTE misschien een beetje overbodig
Given that home users have become a target for criminals, it is clear that companies, by having a much bigger public presence, are an even more attractive target for attacks over the internet.
Financial damages can be incurred in case of downtime, while that is often only a side effect of mitigating a malware infection rather than a direct result of it.
Next to the usual threats of identity theft and stolen login information, corporate networks are also at risk of losing control over their trade secrets.
Although attacks on corporate networks can be expected to use more sophisticated - meaning: not completely automated - methods to gain and maintain access than what the home user is exposed to, the completely automated malware indiscriminately targets home and business computers alike.

%For example, it is possible to buy or rent a botnet to send spam or attack a website with a DDoS attack.
%Some providers of such DDoS services even provide a free trial in the form of a 5-minute DDoS attack as a means to establish legitimacy.

Anti-Virus (AV) programs are advertised as a treatment or inoculation against malware.
By scanning files on the computer's hard drive and comparing them to signatures of known malware or doing heuristic analysis of the code, it is possible to detect and remove infections.

Research has shown that criminal organizations actively develop malware for a specific purpose.
These darker shades of security research employ techniques of obfuscation to remain undetected from AV software, such as self-modifying code, run-time decryption of code and injection of code into other processes.
A common practice in malware spreading is the use of a so-called `packer' or `crypter', software that encapsulates another program in an obfuscated way by loading the latter through a unique (and therefore unknown to AV vendors) `stub' of loader code.
This provides a modular way to employ (and re-use) different types of malware without triggering detection by AV software.

Malware authors and AV vendors are in a continual arms race to outsmart one another, and must monitor each others development attentively to remain effective.
Our research focusses on the competence of AV vendors to keep up with the steady stream of new crypters that are published freely on the internet.
Using a crypter allows us to create a harmless packed binary, which we can then test for detection in various AV packages to assess the AV vendors' vigilance in detecting new crypters and distributing appropriate signatures.

%A tool for malware developers to beat the anti-virus programs is the use of crypters.
%A crypter is software designed to make other software undetectable for an anti-virus program by encrypting it. Therefore it becomes possible to use viruses that would normally be detected. 

%The use of crypters would enable criminal organizations to re-use malware for their financial gain.

 
\subsection{Research question}
We aim to answer to the following research question:

\begin{quote}
What is the effectiveness of AV vendors in detecting new threats?
\end{quote}

\subsubsection{Subquestions}

\begin{quote}
What are the principles behind modern AV software?
\end{quote}

\begin{quote}
What are the techniques used to evade detection by AV software?
\end{quote}

\subsubsection{Hypothesis}
We expect that leading AV vendors will detect and distribute appropriate signatures for a newly published, publicly available crypter in no less than one, but no more than five days.
On average we expect the response and turnover time to be around three days.
This expected delay would be enough time to allow an attacker to ravish whatever is available from a home user's computer.

\section{Methodology}
There is a reasonably successful blog titled `FullyUndetected' that publishes download links to newly released malware and crypters.
The title advertises the fact that these crypters are indeed, in some cases, completely undetected by AV software.
Our aim is to employ at least three of these tools which are advertised as being completely undetected to package a harmless program (calc.exe) to create a detectable `malware' sample on the day these tools are released, and monitor the detection results from various AV programs.
We are aided in this task by the website VirusTotal, which allows users to upload and scan malware samples with over two dozen AV software packages.
The site provides an API to query the detection status of an uploaded sample by its MD5 checksum, so that we can automate the process of collecting results over time, and thereby measure vendors' response time.

We propose querying this information hourly over the course of one month, or until all AV vendors have provided detection capability for the sample, whichever comes first.

\section{Scope}
Although we will acquire tools used for dubious purposes, we will not actually infect any computers with real malware.
We will use a harmless program, packed with a public crypter to create a harmless binary that should, eventually, trigger AV software detection.

\section{Motivation}
I (Aram) only recently became aware of the thriving hobby industry of crypter development, and have been curious to learn how AV vendors respond to these threats ever since.

\newpage
\section{Time schedule}
\textbf{Dit is nog de oude tabel.}
\begin{center}
  \begin{tabular}{ | r | l | l || l | }
    \hline
    Week & Date & Activity & Deliverable \\ \hline
    39 &  & Researching NFC &  \\ \hline
    40 & 8-10 & Researching NFC & Introduction chapter \\ \hline
    41 & 15-10 & Technical analysis, NFC communication & chapter NFC communication \\ \hline
    42 & 22-10 & Technical analysis, hardware architecture & chapter hardware architecture \\ \hline
    43 & & Autumn break & \\ \hline
    44 & 5-11 & Technical analysis, generic software & chapter generic software \\ \hline
    45 & & Case studies, analyse security & \\ \hline
    46 & 19-11 & Case studies, analyse security & chapter case studies\\ \hline
    47 & & Preparing presentation and draft paper & \\ \hline
    48 & & Preparing presentation and draft paper & \\ \hline
    49 & 6-12 & & Draft paper and presentation sheets due \\ \hline
    49 & 7-12 & & Final presentation  \\ \hline
    49 & 9-12 & Supervisor review & \\ \hline
    50 & 17-12 & Feedback due & \\ \hline
    51 & & Christmas break & \\ \hline
    52 & & Christmas break & \\ \hline
    1 & & Revise paper & \\ \hline
    3 & 17-01 & Revise paper & Revised final paper due \\ \hline
  \end{tabular}
\end{center}

%Source language
%Lines of code / binary size
%Encryption methods
%Country of origin
%Motivation
%Target Operating System / execution environment
%Number of known variants


%\begin{itemize}
%\item{Remote Administration Tools/Trojans (RATs)}
%\item{Fake Antivirus}
%\item{Fake Antivirus}
%\end{itemize}

\bibliographystyle{alpha}
%\bibliography{references}
\end{document}
