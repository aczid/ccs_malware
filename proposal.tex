% !TEX TS-program = pdflatex
% !TEX encoding = UTF-8 Unicode

% This is a simple template for a LaTeX document using the "article" class.
% See "book", "report", "letter" for other types of document.

\documentclass[9pt]{article} % use larger type; default would be 10pt

\usepackage[utf8]{inputenc} % set input encoding (not needed with XeLaTeX)
\usepackage{url}

%%% PAGE DIMENSIONS
\usepackage{geometry} % to change the page dimensions
\geometry{a4paper} % or letterpaper (US) or a5paper or....

\usepackage{graphicx} % support the \includegraphics command and options

%%% PACKAGES
\usepackage{booktabs} % for much better looking tables
\usepackage{array} % for better arrays (eg matrices) in maths
\usepackage{paralist} % very flexible & customisable lists (eg. enumerate/itemize, etc.)
\usepackage{verbatim} % adds environment for commenting out blocks of text & for better verbatim
\usepackage{subfig} % make it possible to include more than one captioned figure/table in a single float
% These packages are all incorporated in the memoir class to one degree or another...

%%% HEADERS & FOOTERS
%\usepackage{fancyhdr} % This should be set AFTER setting up the page geometry
%\pagestyle{fancy} % options: empty , plain , fancy
%\renewcommand{\headrulewidth}{0pt} % customise the layout...
%\topmargin = 0pt 
\usepackage[cm]{fullpage}
\textheight = 750pt
\textwidth = 500pt
%\headheight = 10pt
%\lhead{10pt}\chead{10pt}\rhead{10pt}
%\lfoot{10pt}\cfoot{\thepage}\rfoot{10pt}

%%% SECTION TITLE APPEARANCE
%%%\usepackage{sectsty}
%%%\allsectionsfont{\sffamily\mdseries\upshape} % (See the fntguide.pdf for font help)
% (This matches ConTeXt defaults)

%%% ToC (table of contents) APPEARANCE
\usepackage[nottoc,notlof,notlot]{tocbibind} % Put the bibliography in the ToC
\usepackage[titles,subfigure]{tocloft} % Alter the style of the Table of Contents
\renewcommand{\cftsecfont}{\rmfamily\mdseries\upshape}
\renewcommand{\cftsecpagefont}{\rmfamily\mdseries\upshape} % No bold!

%%% END Article customizations

%%% The "real" document content comes below...

\title{Cybercrime Science: Effectiveness of logfile-analysis against botnets}
\author{Mark Vijfvinkel \& Aram Verstegen \\ 4077148 4092368}
\date{} % Activate to display a given date or no date (if empty),
         % otherwise the current date is printed 

\begin{document}
\maketitle

%\abstract{
%We aim to investigate the current trends in malware.
%Focussing on the past five years, we will try to map out the various propagation methods, features, protocols and economics of various malware such as banking trojans, Remote File Inclusion bots, scare- and ransomware.
%By collecting various types of current malware, we will try to find patterns of malware operation and classify them as a means to predict coming trends.
%
%We hope to be able to discover more recent malware during the course of the semester, which will hopefully bring even more variation to the already diverse ecosystem of malware.
%One of the products we hope to generate as an offshoot of our research is a visual representation of the malware trends over time.


%\cite{kanich2008spamalytics}
%}

\section{Problem}
%The conjunction `malware' is a term to cover various families of Malicious Software.
Malware can be classified by its means of spreading, for example (self-replicating) \textit{viruses} - or \textit{worms} - and \textit{trojans} (innocent-looking programs containing a malicious payload).
%A distinction can also be identified between types of payloads, with illustrative titles like \textit{keyloggers}, \textit{adware} and \textit{spyware}.

%Viruses initially manifested as a general annoyance by drawing attention to themselves but have since evolved to become more advanced at remaining stealthy and, for example, surreptitiously exfiltrating the home user's internet banking details in a completely automated way.
The most insidious of such malware would be the \textit{bot} (short for robot). A bot is a program that provides control over the machine to a remote attacker's Command and Control (C\&C) software. By distributing these bots on a massive scale, many computers will become infected and take part in the botnet. These botnets can grow up to thousands or millions of computers, often used to stage a \textit{Distributed Denial of Service} (DDoS) attack or send huge amounts of spam.

%Most users have been exposed to malware at some point.
%Modern malware has an impact on both home and business users, and has become a persistent problem since the worldwide adoption of the internet.
%Internet banking fraud is a serious problem today, and the even more insidious threat of identity theft is on the rise.

% NOTE misschien een beetje overbodig
%Given that home users have become a target for criminals, it is clear that companies, by having a much bigger public presence, are an even more attractive target for attacks over the internet.
%Financial damages can be incurred in case of downtime, while that is often only a side effect of mitigating a malware infection rather than a direct result of it.
%Next to the usual threats of identity theft and stolen login information, corporate networks are also at risk of losing control over their trade secrets.
%Although attacks on corporate networks can be expected to use more sophisticated - meaning: not completely automated - methods to gain and maintain access than what the home user is exposed to, the completely automated malware indiscriminately targets home and business computers alike.

%For example, it is possible to buy or rent a botnet to send spam or attack a website with a DDoS attack.
%Some providers of such DDoS services even provide a free trial in the form of a 5-minute DDoS attack as a means to establish legitimacy.

Anti-Virus (AV) programs are advertised as a treatment or preventative measure against malware.
By scanning files on the computer's hard drive and comparing them to signatures of known malware or doing heuristic analysis of the code, it is possible to detect and remove infections.



%Research has shown that criminal organizations actively develop malware for a specific purpose.
%These darker shades of security research employ techniques of obfuscation to remain undetected from AV software, such as self-modifying code, run-time decryption of code and injection of code into other processes.
%A common practice in malware spreading is the use of a so-called `packer' or `crypter', software that encapsulates another program in an obfuscated way by loading the latter through a unique (and therefore unknown to AV vendors) `stub' of loader code.
%This provides a modular way to employ (and re-use) different types of malware without triggering detection by AV software.

%Malware authors and AV vendors are in a continual arms race to outsmart one another, and must monitor each others development attentively to remain effective.
Our research focusses on a different approach to detect bots and botnet activity, namely logfile analyses. It is possible for a computer to log all the actions it performs. By analysing the logfile for certain entries e.g. log in attempts, certain traffic sent, on regular basis, it might become possible to detect malicious activity that an AV might not pick up. By sending an email to the user or actively intervening e.g. isolate it from the network, action can be taken to remove the bot and therefore shrinking the botnet. By shrinking the botnet it will become harmless. This way it will become possible to prevent a botnet from, growing and stage future attacks and therefore preventing the crime that botnets bring with them.

%Using a crypter allows us to create a harmless packed binary, which we can then test for detection in various AV packages to assess the AV vendors' vigilance in detecting new crypters and distributing appropriate signatures.

%A tool for malware developers to beat the anti-virus programs is the use of crypters.
%A crypter is software designed to make other software undetectable for an anti-virus program by encrypting it. Therefore it becomes possible to use viruses that would normally be detected. 

%The use of crypters would enable criminal organizations to re-use malware for their financial gain.

 
\subsection{Research question}
We aim to answer to the following research question:

\begin{quote}
What is the effectiveness of logfile-analyses against a botnet?
\end{quote}

%\subsubsection{Subquestions}

%\begin{quote}
%What are the principles behind modern AV software?
%\end{quote}

%\begin{quote}
%What are the techniques used to evade detection by AV software?
%\end{quote}

%\subsubsection{Hypothesis}
%We expect that leading AV vendors will detect and distribute appropriate signatures for a newly published, publicly available crypter in no less than one, but no more than five days.
%On average we expect the response and turnover time to be around three days.
%This expected delay would be enough time to allow an attacker to ravish whatever is available from a home user's computer.

\section{Methodology}
We will use a trial version of Splunk, a tool which can analyze machine data one of which is logfiles. We have to install this on a server and use a different machine to infect with malware. The infected machine will be set up to log as much as possible. The logfile will be sent to the Splunk machine, where it will be analyzed by using filters. Hopefully the bot or botnet activity will be detected.
We can expand the experiment in a later stage by using more infected machines.
By comparing the results to results of a machine just running anti-virus, we can see if there is a difference in the effectiveness. Then we could make a recommendation to use only logfile-analysis or use it next to anti-virus.

\section{Risks}
Setting up Splunk might take time, we can mitigate this reading to the manual and estimating the set up time.
Nothing will be detected, this is still a result, but can be mitigated by setting up the filters correctly.
By using filters we are depending on what is known activity for the malware that is tested. This can be mitigated by trying to find a pattern that is used by different bots. 
The malware can spread and cause damage to other equipment. This can be mitigated by using virtual machines and a closed network.


\section{References}


Automated Log Analysis of Infected Windows OS Using Mechanized Reasoning, by Ruo Ando, 2009.
 
Log-Based Distributed Security Event Detection Using Simple Event Correlator, by Justin Myers and Michael R. Grimaila and Robert F. Mills, 2011.

Computer Viruses and Malware, by John Aycock, 2006.


%There is a reasonably successful blog titled `FullyUndetected' that publishes download links to newly released malware and crypters.
%The title advertises the fact that these crypters are indeed, in some cases, completely undetected by AV software.
%Our aim is to employ at least three of these tools which are advertised as being completely undetected to package a harmless program (calc.exe) to create a detectable `malware' sample on the day these tools are released, and monitor the detection results from various AV programs.
%We are aided in this task by the website VirusTotal, which allows users to upload and scan malware samples with over two dozen AV software packages.
%The site provides an API to query the detection status of an uploaded sample by its MD5 checksum, so that we can automate the process of collecting results over time, and thereby measure vendors' response time.

%We propose querying this information hourly over the course of one month, or until all AV vendors have provided detection capability for the sample, whichever comes first.

%\section{Scope}
%Although we will acquire tools used for dubious purposes, we will not actually infect any computers with real malware.
%We will use a harmless program, packed with a public crypter to create a harmless binary that should, eventually, trigger AV software detection.

%\section{Motivation}
%I (Aram) only recently became aware of the thriving hobby industry of crypter development, and have been curious to learn how AV vendors respond to these threats ever since.

%I (Mark) heard about the technique used by crypters a couple of years ago through a friend, but forgot about it. I am mostly interested in the speed with which the AV vendors take action, since it is already "old" malware that is basically attacking the user.

%Source language
%Lines of code / binary size
%Encryption methods
%Country of origin
%Motivation
%Target Operating System / execution environment
%Number of known variants


%\begin{itemize}
%\item{Remote Administration Tools/Trojans (RATs)}
%\item{Fake Antivirus}
%\item{Fake Antivirus}
%\end{itemize}

\bibliographystyle{alpha}
%\bibliography{references}
\end{document}
